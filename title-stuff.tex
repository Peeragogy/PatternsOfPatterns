\title{Patterns of Patterns}

\author{Joseph Corneli}
\authornote{Both authors contributed equally to this research.}
\email{jcorneli@brookes.ac.uk}
\orcid{1234-5678-9012}
\author{Alex Murphy}
\email{jcorneli@brookes.ac.uk}
\affiliation{%
  \institution{Oxford Brookes University}
  \streetaddress{P.O. Box 1212}
  \city{Oxford}
  \country{England, UK}
  \postcode{43017-6221}
}

\author{Raymond S. Puzio}
\authornotemark[1]
\email{rsp@hyperreal.enterprises}
\author{Leo Vivier}
\email{lv@hyperreal.enterprises}
\affiliation{%
  \institution{Hyperreal Enterprises Ltd}
  \streetaddress{1 New Street}
  \city{Musselburgh}
  \country{Scotland, UK}}

\author{Noorah Alhasan}
\affiliation{%
  \institution{University of Texas at Austin}
  \streetaddress{1 New Street}
  \city{Austin}
  \country{USA}}
\email{noorah@utexas.edu}

\author{Charles J. Danoff}
\affiliation{%
  \institution{Mr Danoff’s Teaching Laboratory}
  \city{Chicago}
  \country{USA}}
\email{danoff@yahoo.com}

\author{Vitor Bruno}
\affiliation{%
  \institution{Milestone English}
  \city{Sao Paolo}
  \country{Brazil}}
\email{vitor@yahoo.com}

\author{Paola Ricaurte}
\affiliation{%
  \institution{Tec de Monterray}
  \city{Mexico City}
  \country{Mexico}
}
\email{paola@yahoo.com}

\author{Charlotte Pierce}
\affiliation{%
  \institution{Pierce Press}
  \city{M}
  \country{USA}
}
\email{pierce@yahoo.com}

%%
%% By default, the full list of authors will be used in the page
%% headers. Often, this list is too long, and will overlap
%% other information printed in the page headers. This command allows
%% the author to define a more concise list
%% of authors' names for this purpose.
\renewcommand{\shortauthors}{Corneli et al.}

%%
%% The abstract is a short summary of the work to be presented in the
%% article.
\begin{abstract}
  We introduce Inayatullah's Causal Layered Analysis (CLA) method from
  the field of futures studies into the domain of patterns.  We
  develop a case study that integrates CLA with patterns in
  collaborative research, use the method to understand the way the
  design pattern discourse is evolving, and and discuss implications
  for other research and innovation projects.
\end{abstract}

%%
%% The code below is generated by the tool at http://dl.acm.org/ccs.cfm.
%% Please copy and paste the code instead of the example below.
%%
\begin{CCSXML}
<ccs2012>
 <concept>
  <concept_id>10010520.10010553.10010562</concept_id>
  <concept_desc>Computer systems organization~Embedded systems</concept_desc>
  <concept_significance>500</concept_significance>
 </concept>
 <concept>
  <concept_id>10010520.10010575.10010755</concept_id>
  <concept_desc>Computer systems organization~Redundancy</concept_desc>
  <concept_significance>300</concept_significance>
 </concept>
 <concept>
  <concept_id>10010520.10010553.10010554</concept_id>
  <concept_desc>Computer systems organization~Robotics</concept_desc>
  <concept_significance>100</concept_significance>
 </concept>
 <concept>
  <concept_id>10003033.10003083.10003095</concept_id>
  <concept_desc>Networks~Network reliability</concept_desc>
  <concept_significance>100</concept_significance>
 </concept>
</ccs2012>
\end{CCSXML}

\ccsdesc[500]{Computer systems organization~Embedded systems}
\ccsdesc[300]{Computer systems organization~Redundancy}
\ccsdesc{Computer systems organization~Robotics}
\ccsdesc[100]{Networks~Network reliability}

%%
%% Keywords. The author(s) should pick words that accurately describe
%% the work being presented. Separate the keywords with commas.
\keywords{datasets, neural networks, gaze detection, text tagging}


%%
%% This command processes the author and affiliation and title
%% information and builds the first part of the formatted document.
\maketitle
