\title{Patterns of Patterns}

\author{Joseph Corneli}
\authornote{Both authors contributed equally to this research.}
\email{jcorneli@brookes.ac.uk}
\orcid{1234-5678-9012}
\author{Alex Murphy}
\email{19032710@brookes.ac.uk}
\affiliation{%
  \institution{Oxford Brookes University}
  \streetaddress{Gipsy Lane}
  \city{Oxford}
  \country{UK}
  \postcode{OX3 0BP}
}

\author{Raymond S. Puzio}
\authornotemark[1]
\email{rsp@hyperreal.enterprises}
\author{Leo Vivier}
\email{zaeph@zaeph.net}
\affiliation{%
  \institution{Hyperreal Enterprises Ltd}
  \streetaddress{81 St Clement’s St}
  \city{Oxford}
  \country{UK}
  \postcode{OX4 1AW}}

\author{Noorah Alhasan}
\affiliation{%
  \institution{LBJ School of Public Affairs, The University of Texas at Austin}
  \streetaddress{P.O. Box Y}
  \city{Austin}
  \state{TX}
  \country{USA}
  \postcode{78713-8925}}
\email{noorah.alhasan@utexas.edu}

\author{Vitor Bruno}
\affiliation{%
  \institution{Milestone English}
  \streetaddress{Rua Trieste 170, ap2}
  \city{Palhoca}
  \state{SC}
  \country{Brazil}
  \postcode{88132-227}}
\email{chief@milestoneenglishcourse.com}

% \author{Paola Ricaurte}
% \affiliation{%
%   \institution{Tecnologico de Monterrey}
%   \streetaddress{Calle del Puente 222 Col. Ejidos de Huipulco, Tlalpan}
%   \city{Mexico City}
%   \country{Mexico}
% }
% \email{pricaurt@tec.mx}

\author{Charlotte Pierce}
\affiliation{%
  \institution{Pierce Press}
  \streetaddress{PO Box 206}
  \city{Arlington MA}
  \country{USA}
  \postcode{02476}
}
\email{charlotte.pierce@gmail.com}

\author{Charles J. Danoff}
\affiliation{%
  \institution{Mr Danoff’s Teaching Laboratory}
 \streetaddress{PO Box 802738}
 \city{Chicago}
 \state{IL}
  \country{USA}
  \postcode{60680}}
\email{contact@mr.danoff.org}


%%
%% By default, the full list of authors will be used in the page
%% headers. Often, this list is too long, and will overlap
%% other information printed in the page headers. This command allows
%% the author to define a more concise list
%% of authors' names for this purpose.
\renewcommand{\shortauthors}{Corneli et al.}


% (1) Review the intention: what do we expect to learn or make together?

% - Joe: Wanted to walk through the PoP paper with Rebecca, in order to help solidify my own grasp of the concepts and get her feedback.
% - Rebecca: This topic sounded interesting when you mentioned it and I wanted to learn about Pop

% (2) Establish what is happening: what and how are we learning?

% - Indeed we did speed through the paper, Rebecca pointed out a few places where there was friction with the wording or concepts, like ``PEER-TO-PEER'' and also suggested Operational Research and Strategy as an appropriate topic; mentioned “improvisatory” style
% - Interruptions were welcomed!
% - Rebecca: was hesitant to interrupt the narrative
% - This is a bit different from usual IEAI style...

% (3) What are some different perspectives on what’s happening?

% - Joe: appreciate the time Rebecca has put into this a lot, and I also think this was a good way to present the paper
% - Rebecca: I think you assume knowledge in the presentation, and I think you need to assume the listener (if they aren’t in the area) that they don’t know anything.  It wouldn’t be patronising to explain the basic concepts.

% (4) What did we learn or change?

% - Talking to Karl, he would reconise one of the areas (probably) but not necessarily the other two.  Everyone is going to be new to some of these concepts.
% - This was great as a ``final edit'' — we will also be able to edit this paper until December
% - RR: is it your aim to automate?

% (5) What else should we change going forward?

% - Joe: if it would be helpful to RR, I’d certainly be happy to meet again about these ideas
% - Would this (patterns of patterns) to actually be useful for ethical AI?
% - E.g., rethink in the context of moral machines

%%
%% The abstract is a short summary of the work to be presented in the
%% article.
% distributed peer-to-peer networks
\begin{abstract}
The purpose of this paper is to show how we can combine and adapt
methods from elite training, future studies, and
collaborative design, and apply them to address significant problems
in social networks.  We focus on three such methods:
we use Action Reviews to implement social perception, Causal Layered
Analysis to implement social cognition, and Design Pattern Languages
to implement social action.  We present the results of two studies:
firstly, we use Causal Layered Analysis to explore the ways in which the design
pattern discourse has been evolving.
Secondly, to illustrate the three methods in combination,
we develop a case study, showing how we applied the methods to
bootstrap a distributed cross-disciplinary research seminar.
Building on these analyses, we
elaborate several scenarios for the future use of design patterns in
large-scale distributed collaboration.
Our case study suggests ways in which progress could be made towards realizing these scenarios.
We conclude that the
combination of methods is robust to uncertainty, insofar as they support adaptations as circumstances change, and incorporate diverse perspectives.  In
particular, we show how methods drawn from other domains enrich and
are enriched by design patterns; we believe the analysis will be of
interest to all of the communities whose methods we draw upon.
\end{abstract}

%%
%% The code below is generated by the tool at http://dl.acm.org/ccs.cfm.
%% Please copy and paste the code instead of the example below.
%%
\begin{CCSXML}
<ccs2012>
<concept>
<concept_id>10003456</concept_id>
<concept_desc>Social and professional topics</concept_desc>
<concept_significance>500</concept_significance>
</concept>
<concept>
<concept_id>10011007.10011074.10011075</concept_id>
<concept_desc>Software and its engineering~Designing software</concept_desc>
<concept_significance>300</concept_significance>
</concept>
<concept>
<concept_id>10011007.10011074.10011134.10003559</concept_id>
<concept_desc>Software and its engineering~Open source model</concept_desc>
<concept_significance>300</concept_significance>
</concept>
<concept>
<concept_id>10010405.10010481</concept_id>
<concept_desc>Applied computing~Operations research</concept_desc>
<concept_significance>300</concept_significance>
</concept>
<concept>
<concept_id>10010147.10010341</concept_id>
<concept_desc>Computing methodologies~Modeling and simulation</concept_desc>
<concept_significance>100</concept_significance>
</concept>
</ccs2012>
\end{CCSXML}

\ccsdesc[500]{Social and professional topics}
\ccsdesc[300]{Software and its engineering~Designing software}
\ccsdesc[300]{Software and its engineering~Open source model}
\ccsdesc[300]{Applied computing~Operations research}
\ccsdesc[100]{Computing methodologies~Modeling and simulation}


%%
%% Keywords. The author(s) should pick words that accurately describe
%% the work being presented. Separate the keywords with commas.
\keywords{Design Patterns, Pattern Languages, Action Reviews, Futures
Studies, Causal Layered Analysis, Emacs, Free Software, Peeragogy,
Climate Change, Innovation, Anticipation}


%%
%% This command processes the author and affiliation and title
%% information and builds the first part of the formatted document.
\maketitle
